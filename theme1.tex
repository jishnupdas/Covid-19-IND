% primal
\definecolor{cone}{HTML}{373234}
\definecolor{ctwo}{HTML}{c8d7d2}
\definecolor{ctre}{HTML}{d84339}
\definecolor{cfor}{HTML}{f4f2f0}
\definecolor{cfiv}{HTML}{dddddd}

% setting some colors for the theme
% light themes
\setbeamercolor{palette primary}   {bg=cfor!30,fg=cone!95}
\setbeamercolor{palette secondary} {bg=ctre!80,fg=cfiv!70}
\setbeamercolor{palette tertiary}  {bg=ctwo,fg=cfiv!70}
\setbeamercolor{palette quaternary}{bg=cone!90,fg=cfiv!70}
\setbeamercolor{background canvas} {bg=cfiv!20}
\setbeamercolor{structure}         {bg=ctwo!40,fg=cone}         % itemize, enumerate, etc
\setbeamercolor{alerted text}      {bg=cone!40,fg=ctre}
\setbeamercolor{normal text}       {fg=cone!50!black!85}        % normal text color
\setbeamercolor{section in toc}    {bg=ctwo!70,fg=cone}         % TOC sections
\setbeamercolor{titlelike}{parent=structure,fg=cone,bg=ctwo!30}
\setbeamercolor{block body}{use=structure,fg=cone!80!black!90,bg=white!40!ctwo!10}

\useoutertheme{smoothbars} % Alternatively: miniframes, infolines, split, shadow, tree, smoothtree smoothbars
\useinnertheme{rectangles} % Alternatively: circles rectangles rounded inmargin

\usefonttheme{professionalfonts}


%Setting up colors for code listing
\usepackage{listings} % Code listings
\lstset{ 
  backgroundcolor=\color{gray!10}, % choose the background color; you must add \usepackage{color} or \usepackage{xcolor}; should come as last argument
  basicstyle=\footnotesize,        % the size of the fonts that are used for the code
  breakatwhitespace=false,         % sets if automatic breaks should only happen at whitespace
  breaklines=true,                 % sets automatic line breaking
  captionpos=b,                    % sets the caption-position to bottom
  commentstyle=\color{cone!50!green!90},       % comment style
  deletekeywords={...},            % if you want to delete keywords from the given language
  escapeinside={\%*}{*)},          % if you want to add LaTeX within your code
  extendedchars=true,              % lets you use non-ASCII characters; for 8-bits encodings only, does not work with UTF-8
%   firstnumber=1000,                % start line enumeration with line 1000
  frame=single,	                   % adds a frame around the code
  keepspaces=true,                 % keeps spaces in text, useful for keeping indentation of code (possibly needs columns=flexible)
  keywordstyle=\bfseries\color{cone!90},       % keyword style
  %language=Octave,                 % the language of the code
  morekeywords={*,...},            % if you want to add more keywords to the set
  numbers=left,                    % where to put the line-numbers; possible values are (none, left, right)
  numbersep=5pt,                   % how far the line-numbers are from the code
  numberstyle=\tiny\color{gray},   % the style that is used for the line-numbers
  rulecolor=\color{black},         % if not set, the frame-color may be changed on line-breaks within not-black text (e.g. comments (green here))
  showspaces=false,                % show spaces everywhere adding particular underscores; it overrides 'showstringspaces'
  showstringspaces=false,          % underline spaces within strings only
  showtabs=false,                  % show tabs within strings adding particular underscores
  stepnumber=1,                    % the step between two line-numbers. If it's 1, each line will be numbered
  stringstyle=\color{ctre},        % string literal style
  tabsize=4,	                   % sets default tabsize to 2 spaces
  title=\lstname                   % show the filename of files included with \lstinputlisting; also try caption instead of title
}
